Ninety-eight skilled and elite alpine ski racers from Norway and Sweden (age M = 18.1 years, SD= 2; 40 females, 58 males) participated in the learning experiment (Table \ref{descriptive_skier} for demographics of skiers and table \ref{descriptive_coach} for demographics of the coaches). After completing four trials in the main slalom course (Figure \ref{fig:courseandstrategies}a) during the baseline test, where skiers skied the course as quickly as possible without instructions or feedback, they were allocated into three groups using a randomized-blocked approach to adjust for any preexisting differences in the skiers' performance levels \cite{maxwell_designing_2017}. In their allocated groups, skiers were introduced to the defined strategies with a coach, followed by a familiarization phase where skiers tried to ensure that they understood them and were able to execute them. The next phase involved a 'forced exploration' session in which skiers tested the four strategies twice each, with the order randomly determined for each skier under the condition that the first and last four trials tested comprised all four strategies. This provided insight into the effectiveness of the strategies. After forced exploration, skiers completed two free-choice sessions where the coach (supervised learning) or the skier (reinforcement learning) chose the strategy for each trial. On the final day, skiers completed four trials in the main course followed by four rounds in a transfer course without timing or feedback from a coach. After each session during acquisition (including familiarization), both coaches and skiers ranked the strategies based on their performance. During retention and transfer, only the skiers ranked the strategies.